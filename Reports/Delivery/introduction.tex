%!TEX root = main.tex
\chapter{Introduction}

This document describes the implementation of \emph{Large and Heavily Armoured Warships for Android} (henceforth referred to as \emph{LaHAW}). 




%----------------------------
\section{Background}
%----------------------------

LaHAW has been developed as a part of Group A2's project in the course \emph{TDT4240 – Software Architecture at NTNU}. The goal of this project is to develop a software architecture and an implementation of a game for Android powered devices with focus on specific quality attributes. For the realisation of LaHAW, the attributes chosen were \emph{modifiability} and \emph{usability}.

A second requirement for this project was to implement specific architectural and design patterns. LaHAW implements the \emph{abstract factory} and \emph{MVC} architectural patterns, and the \emph{observer}, \emph{singleton} and \emph{state} design patterns.



%----------------------------
\section{Report structure}
%----------------------------

Chapter \ref{cha:design_details} details the implemented architecture of the final prototype of LaHAW.
The user manual in chapter \ref{cha:users_manual} describes what is required to build the application, how to play the game, and a guided tour detailing a typical game all the way from starting the application to winning a game against the AI.
After the game prototype was deemed ready for it, an acceptance test was performed. The findings of these tests are listed in chapter \ref{cha:test_report}, detailing how the application stack up against our functional and quality requirements.
In chapter \ref{cha:relationship_with_the_architecture}, we will detail the inconsistencies between the architecture and the actual implementation of LaHAW.
At the end of this document, in chapter \ref{cha:problems_issues_and_points_learned}, the different problems and issues that emerged during the realisation of LaHAW are discussed, as well as a section describing what we have learned from this project as a whole.