%!TEX root = main.tex
\chapter{Introduction}
This document contains the requirements for an Android game called "Large and Heavily Armored Warships" (LaHAW) , to be created by Trond Kjetil Bremnes, Marius Glittum, Alexander Perry, Even Stene, Elisabeth Solheim and Trond Klakken. The purpose of the creation of the game is to train ourselves in use of \emph{Commercial Of The Shelf} (COTS) software, as well as architechtural design.

The functional requirements that has been decided upon in creating the game is discussed in its own chapter. Following this, the quality requirements decidied upon is similarily discussed.

The different commercial off-the-shelf components that has been utilised in realising the application is discussed in further detail in the "COTS" chapter.

% TODO - litt usikker på hva som ligger i disse issuesene. Mest sannsynlig er det ting som _kan_ skje, og våre strategier for a mitigere dette. Men kanskje ikke.
The various issues and problems that has arisen in the course of the project ...
are discussed in the "Issues" chapter.

Lastly, this document will list the different revisions that it has been through during the develpment of the application. This revision history is found in the "Changes" chapter.



\section{The game}
\textsc{Large and Heavily Armored Warships (LaHAW)} is based on the popular Battleships\cite{battleship} game, a game where the objective of the game is to sink enemy ships, while conceiling the positions of ones own ships. The main game screen is an ocean space with multiple tiles (like a grid), where each tile represents a coordinate on the player's ocean space. A ship can span several tiles, in the x and y direction (or it could use just one tile). 
\\


\begin{tabular}{| c | c | c |}
    \hline
    \rowcolor[gray]{0.8}
    \hspace{0.3cm}\textbf{Ship name}\hspace{0.3cm} & \textbf{No. of occupied tiles} & \textbf{No. of ships per player} \\
    \hline
    Aircraft & 5 & 1 \\
    Battleship & 4 & 2 \\
    Submarine & 3 & 2 \\
    Destroyer & 3 & 1 \\
    Patrol boat & 2 & 2 \\
    \hline
\end{tabular}
\label{shiptable}





\section{Goals}
The goals of this project is to create a playable prototype of the game discussed above. The game itself is however not the main goal of this project, but rather the process used in realising it as well as the architectural implementation of the application code.
