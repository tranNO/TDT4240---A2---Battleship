\chapter{Quality Requirements}
Below we will detail the two main quality attributes for the application: \emph{modifiability} that was mandatory for the application; and \emph{usability} that we chose for ourselves. 


%---------------
% Modifiability
%---------------
\section{Modifiability}
Modifiability in an architectural sense is concerned about the cost of change \cite{pensum}. Our main approach is mainly concerned with changes made in \emph{design time}\footnote{Design time is conserned with measures and initiatives that is being implemented as the application is being designed. Even though everythign leading up the moment of delivery is in a broad sense design time, most of the initiatives should be implemented in the desing phase of the project. This is opposed to \emph{run time}, which is conserned with what occurs after the application has been deployed and is being used by an actual end user.}.

% TODO: KILDE!

    \begin{itemize}
        \item[\textbf{M1}] Change difficulty. \\
        \textit{\small{The user is able to choose between three difficulty levels: easy; medium; and hard. Ocean space is determined by these, where the easy setting gives the largest ocean space, and hard setting gives the smallest ocean space. If the ocean space is small, the probability to get hit increases and vice versa. The opposing players' difficulty settings are asymmetrical, such that a player playing with the difficulty setting \emph{easy}, can play against a player with the difficulty setting \emph{hard}.}}
        
        \begin{tabular}{| l | l |}
            \hline
            \rowcolor[gray]{0.8}
            \textbf{Portion of scenario} & \textbf{Values} \\
            \hline
            Source &  Developer \\
            Stimulus & Add difficulty opportunity \\
            Artifact & A change on target system \\
            Environment & Design time \\
            Response & Opportunity to change ocean space  \\
            Response measure & 10 hours to modify  \\
            \hline
        \end{tabular}
        
        \item[\textbf{M2}] Set player color. \\
        \textit{\small{The user should be able to change the color of their player color, i.e. the colors of their ships and other UI elements.}}
        
        \begin{tabular}{| l | l |}
            \hline
            \rowcolor[gray]{0.8}
            \textbf{Portion of scenario} & \textbf{Values} \\
            \hline
            Source & Developer \\
            Stimulus & Add opportunity to change your player color \\
            Artifact & A change on target system \\
            Environment & Design time \\
            Response & Opportunity to change ship colors  \\
            Response measure & 10 hours to modify \\
            \hline
        \end{tabular}
    
        \item[\textbf{M3}] Set player name. \\
        \textit{\small{The user should be able to set their player name before they start the game.}}
        
        \begin{tabular}{| l | l |}
            \hline
            \rowcolor[gray]{0.8}
            \textbf{Portion of scenario} & \textbf{Values} \\
            \hline
            Source & Developer \\
            Stimulus & Add opportunity to set your player name \\
            Artifact & A change on target system \\
            Environment & Design time \\
            Response & Opportunity to set player name  \\
            Response measure & 5 hours to modify \\
            \hline
        \end{tabular}
    \end{itemize}



%---------------
% Usability
%---------------
\section{Usability}
According to Bass, Clements and Kazman, \emph{usability} is concerned about making the the tasks involved in the system as easy to accomplish as possible \cite[p.~90]{pensum}. Below we have identified the following use cases for assuring the usability of LaHAW.

    \begin{itemize}
        \item[\textbf{U1}] Placing the ships. \\
        \textit{\small{The user should be able to see each of their ships being placed on the ocean.}}
        
        \begin{tabular}{| l | l |}
            \hline
            \rowcolor[gray]{0.8}
            \textbf{Portion of scenario} & \textbf{Values} \\
            \hline
            Source & End User \\
            Stimulus & Change position of the ships \\
            Artifact & System \\
            Environment & Game play \\
            Response & Move ships  \\
            Response measure & After positioning the ships, the user is able to change these positions, one at a time. \\
            \hline
        \end{tabular}

        \item[\textbf{U1}] Communicating legal actions \\
        \textit{\small{The user should be able to see each of their ships being placed on the ocean.}}
        
        \begin{tabular}{| l | l |}
            \hline
            \rowcolor[gray]{0.8}
            \textbf{Portion of scenario} & \textbf{Values} \\
            \hline
            Source & End User \\
            Stimulus & Change position of the ships \\
            Artifact & System \\
            Environment & Game play \\
            Response & Move ships  \\
            Response measure & The system shall prevent the user from trying to place a ship in an illegal position. \\
            \hline
        \end{tabular}

        \item[\textbf{U2}] Player name and color saved. \\
        \textit{\small{The player name and color will be saved at start for later use in the game.}}
        
        \begin{tabular}{| l | l |}
            \hline
            \rowcolor[gray]{0.8}
            \textbf{Portion of scenario} & \textbf{Values} \\
            \hline
            Source & End User \\
            Stimulus & Store player name and color \\
            Artifact & System \\
            Environment & At configure time \\
            Response & The player name will appear at the game play screen, \\
             & and the ships will be painted in the predefined player color  \\
            Response measure & Name and color updates when the player starts the game \\
            \hline
        \end{tabular}

        \item[\textbf{U3}] Dialog boxes confirming the users actions. \\
        \textit{\small{The user should be promted to confirm their actions for severe decisions.}}
        
        \begin{tabular}{| l | l |}
            \hline
            \rowcolor[gray]{0.8}
            \textbf{Portion of scenario} & \textbf{Values} \\
            \hline
            Source & End User \\
            Stimulus & User regret their decision \\
            Artifact & System \\
            Environment & Run-time \\
            Response & Dialog box with confirmation button(s) (i.e. "OK") \\
            Response measure & Goes to either the next step in the process, or the previous step \\
            \hline
        \end{tabular}
        
        \newpage
        
        \item[\textbf{U4}] No action should be more than three clicks away. \\
        \textit{\small{To increase efficiency.}}
        
        \begin{tabular}{| l | l |}
            \hline
            \rowcolor[gray]{0.8}
            \textbf{Portion of scenario} & \textbf{Values} \\
            \hline
            Source & End User \\
            Stimulus & User wants to use the system efficiently \\
            Artifact & System \\
            Environment & Run-time \\
            Response & Support efficient use \\
            Response measure & Reduce task time \\
            \hline
        \end{tabular}
        
        \item[\textbf{U5}] The game should show game hints whereever appropriate. \\
        \textit{\small{The user gets well-founded recommendations or tips during game play, in order to increase confidence.}}
        
        \begin{tabular}{| l | l |}
            \hline
            \rowcolor[gray]{0.8}
            \textbf{Portion of scenario} & \textbf{Values} \\
            \hline
            Source & End User \\
            Stimulus & User is uncertain on how the game is played, \\
             & or what their next move should be \\
            Artifact & System \\
            Environment & Run-time \\
            Response & A message box with tips \\
            Response measure & User can play the game without problem regarding to game control in \emph{1 hour}\\
            \hline
        \end{tabular}
    \end{itemize}