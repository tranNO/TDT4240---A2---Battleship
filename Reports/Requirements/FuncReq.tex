%!TEX root = main.tex
\chapter{Functional Requirements}
% What is functional requirement?
Sumtin

% What is priority?
Each of the requirements are prioritised according to how important they are for the project as a whole with a three step scale.

\begin{itemize}
    \item{H}
    \item{M}
    \item{L}
\end{itemize}

For a small project such as this, we feel that the three step scale gives us enough data for decisons without sacrificing too much fidelity. While the gap between M and H is fairly large, we have no need for the ability to differentiate between a 76 and 77 (on a 1-100 scale).

In addition to these, the requirements will also be prioritised according to how hard the requirements are to fulfill. These priorities will use the same L-M-H scale as descibed above.

What you then end up with is a way to perform a \emph{cost/benefit} analysis for each of the requirements. A requirement that is hard to implement properly, but is of little importance to the project as a whole could be dropped in benefit for a more important one (or a less difficult one), if time and/or costs involved are deemed to be to large.


% What is love?
% ♥

\section*{FR1 – Change difficulty/size of ocean space}
The user should be able to choose between different difficulty levels. When a level is chosen, the game creates the grid with a given size.

\section{FR2 – Set/change player name}
The player should be able to change their name.

\section{FR3 – Game over}
The game ends if a player gets all of their ships destroyed. The winner is the player with any ships left.

\section{FR4 – Place ones ships at the start of the game}
The player should be able to place his/her ships at the start of the game and only then. During gameplay the ships must be stationary. He/she can choose to use any of the ships given at start, as seen in the table in chapter \ref{shiptable}.

\section{FR5 – Play audio}
Play audio when a ship is hit and when the player misses.

% Bakgrunnsmusikk, som bestemt av ShareLateX – http://www.youtube.com/watch?v=bkysjcs5vFU

\section{FR6 – Hit enemy ships}
A players must be able to fire a shot on his/her turn onto one of the tiles that have not fired upon.

\section{FR7 – Register hits on friendly ships}
A player should be able to observe if the enemy player hits one of his/her ship and thus knowing the "score".