%!TEX root = main.tex
\chapter{Stakeholders}
\label{cha:stakeholders}
For this project, the following main stakeholders has been identified: the \emph{end user}; the \emph{developers}; the \emph{ATAM evaluator}; and the \emph{course staff}. We have deemed the \emph{maintainers} stakeholder group as being relevant, even though the code base is to be abandoned after this course has ended. This is due to the fact that the code produced is to be written as if it was to be maintained and expanded upon.

% TODO: Strukturere denne teksten litt bedre
    
    \section{End user}
    The end user of this software will be the players of the game. To make the game appeal to these, the following attributes has been identified.

        \subsubsection*{Playability}
        The game is to understand and play by the end user. Instructions on how to play the game should therefore be easy to read and understand.

        \subsubsection*{Modifiability}
        The player should be able to customise its name and the color of its ships at run time.
    
    
    \section{Developer}
    The developers of this software is the students of the TDT4240 course.

        \subsubsection*{Buildability}
        The product needs to be finished within a short time frame.
        
        \subsubsection*{Modifiability}
        The product should be easy to extend. This is done with extensive use of standard patterns to make the code easy to understand.

        \subsubsection*{Testability}
        The product should be easy to test and verify to make sure that the application is ready for use.

        \subsubsection*{Grade}
        The product and associated reports should be of a high enough quality to warrant a good grade.

    \section{Maintainers}
    The maintainers are the programmers intended to fix, repair and extend the code after the game's delivery. To make the game more maintainable the following attributes has been identified.

        \subsubsection*{Readability}
        The code needs to be easy to read and understand to help the maintainers know how to repair and extend the code.

        \subsubsection*{Maintainability}
        The code base is to be produced to make the code extendable by separating game logic, model data and view data.

        \subsubsection*{Testability}
        To make repairs and extensions testable, the code to be extended has to be testable.

    
    
    \section{ATAM evaluator}
    As part of the project, the preliminary reports are to be audited by an external ATAM group. These auditors are students attending the same course as us.

        \subsubsection*{Reviewability}
        Well written documentation should be used as the basis for an architectural review. Auditors want the system to be as easy to test as possible, to ensure that all features can be tested. Good testability will therefore result in less work for the testers, and in turn a lower cost.


    \section{Course staff}
        \subsubsection*{Reviewability}
        The course staff will evaluate our final product according to certain criteria like how good our documentation is, how easy it is to read and how easy it is to set up and run our product.
