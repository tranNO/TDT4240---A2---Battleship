\chapter{Architectural drivers}
The group has identified the following architectural drivers for the project: \emph{the experience of the developers}; \emph{need for a short time-to-market}; and \emph{the targeted market}. These architectural drivers are the major quality attributes which influences the system's architecture, and in turn the system's architectural tactics\cite{pensum}. Below are our approaches to each of these drivers.

% Architectural focuses on keeping complexity as low as possible, that is avoid unnecessary complexity. 
%Business qualities are non-software system qualities that influence other quality attributes, we have identified the following business qualities.

	\section{Developers' experience}
	The project will be developed with the group members having little to no experience with Android game development. For the project to be successful, the game will be realized by early creating a functioning code base consisting of critical functionality, which then will be expanded upon iteratively.

	\section{Short time to market}
	The product needs to be finished and deliverable by the end of the course duration, which means that the game needs to be quick and easy to develop. As mentioned above, the game will be developed iteratively, making a playable game available as early as possible, yet expandable if time allows for it.

	\section{Market}
	\emph{LaHAW} will be targeted towards youth and young adults, and is to be used on devices with fairly small screens ranging from 3 to 5 inches. Because of this, the game needs to be easy to understand and use, and we have therefore chosen usability as a primary focus for this project. Furthermore, for a game to be addictive and fun, it needs to be as intuitive as possible.

%	\section{Portability}
%	Future development could include conversion of the game to Apple iOS and Windows Phone 7 platforms for market expansion.