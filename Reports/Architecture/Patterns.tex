%!TEX root = main.tex
\chapter{Architectural Patterns}
The following chapter is concerned with the patterns used in realizing LaHAW.

%-------------
%   MVC
%-------------

    \section{Model-View-Controller Pattern}
    \emph{Large and Heavily Armored Warships} shall implement the \emph{Model-View-Controller} (MVC) pattern, an \emph{architectural} pattern, where the program code is divided into three separate compartments, the \emph{Model}, the \emph{View} and the \emph{Controller}. The \emph{Model} will contain the code that keeps track of data relevant to the \emph{View}, data which is manipulated by the \emph{controller}.
    
        \subsection{Implementation}
        Each of the game world's objects and game data\footnote{Including the players, their scores, ships and oceans.} constitutes the \emph{Models} in this architecture. These models keeps track of the present status of each of these elements. Updates to these elements is handled by the controller.
    
        The \emph{View} contains the code that draws the game world as it is dictated by the \emph{Models}.
    
        The \emph{Controller} handles the calculation of the game data. This includes the handling of events (including timers and player input), keeping track of turns and modifying the game's objects.
        Whenever an event occurs, the controller shall interpret this event and do the neccesary calculations and manipulate the game's models, and then notify the view that the game world has changed and needs to be redrawn.
        An alternative implementation of \emph{MVC} could use separate controllers for each of the game's models, but seeing as the game is rather small and that the controller will mostly be controlling the states of the classes, this shall not be done in this implementation.

%-------------
%   SINGLETON
%-------------

    \section{Singleton Pattern}
    \emph{Large and Heavily Armored Warships} shall implement the \emph{Singleton} Pattern, a design pattern used to restrict the instantiation of a class to one object\cite{singleton}.
    Use of the Singleton pattern assures that such classes can only have \emph{one} instance which is globally available to those in need of it and its methods.
    
        \subsection{Implementation}
        The pattern will be implemented in LaHAW wherever only a single instance of a class is needed, and this class is needed globally at some point in time. Use of singletons in the LaHAW code base will however be limited as much as possible, due to it making unit testing more difficult\cite{singleton_unit_testing}.
    
        \begin{itemize}
            \item The game itself shall be implemented as a singleton.
            \item The game scoreboard shall be implemented as a singleton.
            \item The \emph{states} of the objects, as discussed below, are to be implemented as singletons.
        \end{itemize}
    
    
%--------------------
%    ABSTRACT FACTORY
%--------------------

\section{Abstract Factory Pattern}
\emph{Large and Heavily Armored Warships} shall implement the \emph{Abstract Factory} Pattern, a design pattern that provides an interface for creating families of related objects without specifying their concrete classes\cite{abstractfactory}. This pattern is useful for creating similar objects which will be in frequent use.


    \subsection{Implementation}
    The pattern will be implemented in LaHAW with the different ships. The ships could be similar but possess different properties. We implement an abstract factory which creates concrete classes, and let the client code be unaware of how the implementation is done. All we need is an object implementing that abstract factory. This allows objects to implement different types of that factory, in our case this will be different types of ships that share properties, but also is different from the other shiptypes.

%--------------------
%    STATE PATTERN
%--------------------

    \section{State Pattern}
%    State Pattern is a behavioural \emph{design pattern}
    
%    "Allow an object to alter its behaviour when its internal state changes. The object will appear to change its class."  
    With the use of the State design pattern, the objects that constitutes the game can exist in exactly \emph{one} out of a number of internal states, whose change alter the behaviour of the objects.\cite{online:statepattern}
    
    
        \subsection{Implementation}
        The way the state pattern will be implemented in LaHAW is described below.
    
        \begin{itemize}
            \item The game itself exists in one of the following states: \emph{preparation}, where the players are placing their ships on their playing field; \emph{playing}, where the players takes turn in guessing where the opposing player has placed their ships; \emph{paused}, where the game is paused; \emph{game over}, where the game has ended and a winner has been chosen; and \emph{intro}, which consists of the period between the game application has been started and the game is to be started\footnote{Here the number of players, game difficulty settings, etc. is chosen.}.

            \item When the game is in the \emph{playing} state the players will swap between two states depending on : \emph{fire}; and \emph{observe}.
    
            \item The players of the game will have a set number of ships, each of which is occupying a set number of the ocean's tiles depending on the ship's size. Each of the tiles can thus be in one of two states: \emph{occupied} or \emph{vacant}.
    
            \item The ships are consisting of tiles. The amount of tiles a ship consists of is dependent on the ship's type (which in turn dictates the ship's size). Each of these tiles can be in one of two states: \emph{healthy} or \emph{destroyed}.
    
            \item The ships themselves can exist in one of two states: \emph{floating} or \emph{sunk}.            
    
        \end{itemize}
    
        The states themselves shall be implemented as \emph{Singletons}, as discussed above.
    
    
%----------------------
%   OBSERVER PATTERN
%----------------------
    
    \section{Observer Pattern}
    The final pattern LaHAW shall adopt is the Observer design pattern wherever one-to-many messages are to be sent between the classes and methods of the game. Whenever a state observed by a number of classes, state changes are to be broadcasted to these listeners so that each of these are kept up to date on this change
    
        \subsection{Implementation}    
        Below is a description on how the Observer pattern will be implemented in LaHAW.
    
        \begin{itemize}
            \item When the game controller gets an update that tells it that ship X has been hit, this message is then broadcasted to all of the listeners. This includes the players and the methods that draws the ships.
        \end{itemize}
    

    