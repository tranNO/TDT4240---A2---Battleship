\chapter{Introduction}
This document will provide a documentation of the architecture for \textsc{Large and Heavily Armored Warships} (LaHAW), a game developed for Android. The game is developed as a part of the TDT4240 course at NTNU.

This document will focus on architectural design choices and how we satisfy the quality requirements set. Using the IEEE 1471 standard for architectural descriptions of software-intensive systems\cite{IEEE1471} we describe architectural tactics and patterns based on the defined architectural drivers and stakeholders. A description of the 4+1 architectural viewpoint\cite{kruchten} and a rationale for our chosen solution is also provided, and how quality and functional requirements influence the choice of architecture.

Patterns used for the realization of LaHAW is Model-View-Controller as an architectural pattern; and Singleton, Observer and State as design patterns. This document will detail the reasoning behind these choices.

	\section{The game}
	LaHAW is a remake of the classical "Battleship" game\cite{battleship}. Battleship is a turn based guessing game situated on an grid based ocean space. The game is separated into two parts: an introductory preparation state, where both of the players places their ships on their own ocean grid; and the guessing state, where the players takes turn in guessing where the opposing player has placed his or her ships. A correct guess is effectively noted as a 'hit'. Depending on the size of the ship, several hits might be necessary to sink the ship. The game ends after a player has hit and sunk all of the opposing player's ships.